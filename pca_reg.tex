\documentclass[UTF8]{ctexart}
\usepackage{graphicx}
\usepackage{multirow}
\usepackage{booktabs}
\usepackage{latexsym}
\usepackage{indentfirst}
\setlength{\parindent}{2em}
\usepackage{color}
\definecolor{lbcolor}{rgb}{0.9,0.9,0.9}
\usepackage{listings}
\lstset{backgroundcolor=\color{lbcolor}}
\lstset{keywordstyle=\color[rgb]{0,0,1}}
\lstset{commentstyle=\color[rgb]{0.133,0.545,0.133}}
\lstset{stringstyle=\color[rgb]{0.627,0.126,0.941}}
\lstset{language=Matlab}
\lstset{numbers=left}
\lstset{breaklines=true}
\author{何舜成}
\title{主成分分析在线性回归中的作用}
\begin{document}
\maketitle
\section{PCA与线性回归去病态的关系}
PCA是一种常见的数据降维、提取特征的方法,主要思想是消除不同变量之间的相关性,忽略影响较小的变量,从而消除一些冗余的,或者不重要的数据。当PCA应用在数据压缩上时,是一种有损压缩,不能精确还原原始数据。\par
PCA本质上是从数据的协方差矩阵入手,将协方差矩阵$\Sigma=X^{T}X/(n-1)$作正交分解,得\par
\begin{equation}
\Sigma=USU^{T}, where  U^{T}U=I_{n}
\end{equation}
\par
剔除较小的特征值,将$n$维数据降维为$m$维,并得到变换矩阵$U_{m}$,使得\par
\begin{equation}
Z=U^{T}_{m}X
\end{equation}
\par
$Z$即为降维后的数据,保留了大部分信息,其误差由舍去的特征值与所有特征值之和的比值决定。压缩前应当设定该误差限。\par
多元线性回归中去除病态(亦即去除线性相关变量)是也是考虑$X^{T}X$的特征值,去除绝对值较小的特征值(例如小于0.1的特征值),以免在求逆时不稳定甚至无法求逆。\par
可以看到,PCA压缩数据和多元回归中病态的去除方法是相同的,都是去除协方差矩阵中的小特征值,只是目的分别是降低数据维度和去除线性相关。所以可以通过PCA求解病态线性回归问题。
\section{解题步骤}
依照以上分析,可以得到以下计算步骤:\par
(1)读入数据,确定$X$和$Y$,并将其分别作规范化处理,使其成为均值为0,标准差为1的数据;\par
(2)检验协方差矩阵$X^{T}X/(n-1)$的特征值,是否存在线性关系;\par
(3)对数据进行PCA压缩降维,去除线性相关;\par
(4)对降维后的数据进行线性回归;\par
(5)进行F检验,计算置信区间;\par
(6)得到最终表达式。
\section{计算结果}
以下检验均在显著性水平0.05下进行。\par
回归方程及置信区间:\par
\begin{eqnarray*}
y&=&19.5632-3.7900\times 10^{-4}x_{1}-2.1670\times 10^{-6}x_{2}-0.0010x_{3}+0.6070x_{4}\\
& &+0.6799x_{5}-4.1473\times 10^{-4}x_{6}+0.3305x_{7}+2.5180\times 10^{-4}x_{8}+0.1639x_{9}\\
& &+4.3848\times 10^{-4}x_{10}-0.0960x_{11}+0.1540x_{12}+0.0554x_{13}-0.0309x_{14}\\
& &\pm 10.7150
\end{eqnarray*}
\par
$F$检验结果表明可认为线性关系成立:\par
\begin{equation}
F=324.7743>F_{\alpha}=1.8829
\end{equation}
\par
PCA压缩选取的相对误差阈值为0.1,数据压缩率$\eta=64.6\%$。\par
可以看到,最终的turnout与三个人口数据负相关,与老年人人口正相关,与犯罪率负相关,与受高等教育比例正相关,与收入正相关,与农业人口正相关,与民主党投票率正相关,与共和党投票率负相关,与Perot投票率正相关,与白人比例正相关,与黑人比例负相关。\par
这些因素里有许多是有关联的,如白人黑人的比例、犯罪率与受高等教育比例、人口数与人口密度,因此直接用所有因素作线性回归是不可取的,需要去除线性相关变量,这就是PCA在这其中的作用。
\end{document}