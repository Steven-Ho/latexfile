\documentclass[UTF8]{ctexart}
\usepackage{graphicx}
\usepackage{multirow}
\usepackage{booktabs}
\usepackage{latexsym}
\usepackage{indentfirst}
\setlength{\parindent}{2em}
\usepackage{color}
\definecolor{lbcolor}{rgb}{0.9,0.9,0.9}
\usepackage{listings}
\lstset{backgroundcolor=\color{lbcolor}}
\lstset{keywordstyle=\color[rgb]{0,0,1}}
\lstset{commentstyle=\color[rgb]{0.133,0.545,0.133}}
\lstset{stringstyle=\color[rgb]{0.627,0.126,0.941}}
\lstset{language=Matlab}
\lstset{numbers=left}
\lstset{breaklines=true}
\author{何舜成,厉丹阳}
\title{利用摄动理论判断被动行走极限环稳定性}
\begin{document}
\maketitle
\section{摄动理论简介}
在一些很难求得精确解的数学方程中,引入微扰项,从而求得近似的解析解,这种方法称作摄动方法或摄动理论(Perturbation Theory)。下面是一个简单的运用摄动理论求解方程的例子。\par
考虑如下二次方程\par
\begin{equation}
x^{2}-2\epsilon x-1=0, \epsilon\ll1
\end{equation}
\par
由求根公式可以得到精确解如下\par
\begin{equation}
x=\epsilon\pm\sqrt{1+\epsilon^{2}}
\end{equation}
\par
由于$\epsilon\ll1$,可以进行展开\par
\begin{eqnarray*}
x&=&\epsilon\pm(1+\frac{\epsilon^{2}}{2}-\frac{\epsilon^{4}}{8}+\cdots)\\
&=&\pm1+\epsilon\pm\frac{\epsilon^{2}}{2}\mp\frac{\epsilon^{4}}{8}+O(\epsilon^{6})
\end{eqnarray*}
\par
取$\epsilon=0.1$,方程的正根依次是\par
\begin{eqnarray*}
x_{0}&=&1\\
x_{1}&=&1.1\\
x_{2}&=&1.105\\
x_{3}&=&1.1049875
\end{eqnarray*}
\par
其9位精确解为$x=1.10498756$,可以看到随着阶次的提高,近似解收敛到精确解上。随着$\epsilon$趋于0,收敛速度加快。\par
在实际应用中,需要对变量进行尺度变换,以找到一个小量$\epsilon$,将精确解写为$\epsilon$的级数形式,截取前几项作为精确解的近似表达。例如将精确解$A$写成如下级数表达式:\par
\begin{equation*}
A=\epsilon^{0}A_{0}+\epsilon^{1}A_{1}+\epsilon^{2}A_{2}+\cdots
\end{equation*}
\section{跨步方程模型的建立}
McGeer的“Passive Dynamic Walking”一文对经典的被动行走机器人做出了建模,下面简略作介绍。\par
\subsection{Start- to End-of-Step Equation}
第一个方程是一步从开始到结束的状态方程,其中状态变量既有两条腿的角度$\Delta\theta_{C}$和$\Delta\theta_{F}$(表示偏离斜坡表面法方向的角度),合为向量$\Delta\theta$;又有两条腿的角速度$\Omega_{C}$和$\Omega_{F}$,合为向量$\Omega$。一步所用规范化时间为$\tau=t\sqrt{g/l}$。$\Delta\theta_{SE}$表示机器人在斜坡上处于静态平衡时的两条腿角度。经过一些推算(略去,详见McGeer原文),可得\par
\begin{equation}
\left[
\begin{array}{c}
\Delta\theta(\tau_{k})\\
\Omega(\tau_{k})
\end{array}
\right]
=D(\tau_{k})
\left[
\begin{array}{c}
\Delta\theta_{k}-\Delta\theta_{SE}\\
\Omega_{k}
\end{array}
\right]
+
\left[
\begin{array}{c}
\Delta\theta_{SE}\\
0
\end{array}
\right]
\end{equation}
\par
考虑支撑腿转换时两条腿的角度互为相反数,定义\par
\begin{equation}
\lambda=
\left[
\begin{array}{c}
-1\\
1
\end{array}
\right]
\end{equation}
\par
那么有\par
\begin{equation}
\Delta\theta_{k}=\lambda\alpha_{k}
\end{equation}
\begin{equation}
\Delta\theta(\tau_{k})=-\lambda\alpha_{k+1}
\end{equation}
\par
其中$\alpha$是摆动腿触地时腿的角度。
\subsection{Support Transfer}
第二个方程是支撑腿转换时发生非弹性碰撞所导致的。$M^{-}$和$M^{+}$表示支撑腿转换前和转换后的“惯性矩阵”,并定义\par
\begin{equation}
F=
\left[
\begin{array}{cc}
0 & 1 \\
1 & 0
\end{array}
\right]
\end{equation}
\par
那么得到如下方程:\par
\begin{equation}
\Omega_{k+1}=M^{+}^{-1}M^{-}F\Omega(\tau_{k})\equiv\Lambda\Omega(\tau_{k})
\end{equation}
\subsection{``S-to-S" Equations}
结合(3)(5)(6)(8),并将$4\times4$矩阵分解成四个$2\times2$矩阵,可以得到整个周期内离散状态方程表达式。\par
\begin{equation}
-\lambda\alpha_{k+1}=D_{\theta\theta}[\lambda\alpha_{k}-\Delta\theta_{SE}]+D_{\theta\Omega}\Omega_{k}+\Delta\theta_{SE}
\end{equation}
\begin{equation}
\Omega_{k+1}=\Lambda D_{\Omega\theta}[\lamdba\alpha_{k}-\Delta\theta_{SE}]+\Lambda D_{\Omega\Omega}\Omega_{k}
\end{equation}
\section{摄动理论在稳定性分析中的应用}
在状态方程建立之后,McGeer用牛顿法找到了极限环参数的解$\tau_{0}$、$\alpha_{0}$。在极限环找到之后,下一步是分析极限环的稳定性。假设在稳定步态下发生了微小扰动($\tau_{k}-\tau_{0}$和$\alpha_{k}-\alpha_{0}$),那么过渡矩阵$D$和支撑腿转移矩阵$\Lambda$可以由下式估计:\par
\begin{equation}
D(\tau_{k})\approx D(\tau_{0})+\frac{\partial D}{\partial\tau}(\tau_{k}-\tau_{0})
\end{equation}
\begin{equation}
\Lambda(\alpha_{k})\approx\Lambda(\tau_{0})+\frac{\partial\Lambda}{\partial\alpha}(\alpha_{k}-\alpha_{0})
\end{equation}
\par
将(11)(12)式代入(9)(10),得到跨步方程的估计形式\par
\begin{equation}
\left[
\begin{array}{c}
\alpha_{k+1}-\alpha_{0}\\
\Omega_{Ck+1}-\Omega_{C0}\\
\Omega_{Fk+1}-\Omega_{F0}\\
\tau_{k}-\tau_{0}
\end{array}
\right]
=S
\left[
\begin{array}{c}
\alpha_{k}-\alpha_{0}\\
\Omega_{Ck}-\Omega_{C0}\\
\Omega_{Fk}-\Omega_{F0}\\
\end{array}
\right]
\end{equation}
\par
矩阵$S$上$3\times3$方块可以指示极限环的稳定性,若该方阵的特征值全部在单位圆内,那么该极限环是稳定的,否则是不稳定的,在这种情况下,该方程可以用来设计控制器来使步态变得稳定。\par
在极限环判稳定性的问题中,摄动理论起到了对跨步方程线性化的作用,用零阶和一阶项估计非线性项。因此在这里只是摄动理论的一个简单应用。
\end{document}