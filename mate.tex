\documentclass[UTF8]{ctexart}
\usepackage{indentfirst}
\setlength{\parindent}{2em}
\usepackage{cite}
\author{自26 何舜成}
\title{对“阿里巴巴”在美国被告事件\\的看法和对策}
\begin{document}
\maketitle
\section{事件还原}
事件源于阿里巴巴与工商总局的一番博弈。2015年1月27日,淘宝官微发表长微博“一个80后淘宝网运营小二心声”,质疑工商总局几天前发布的2014下半年网络交易商品监测程序有问题。次日,作为回应,工商总局官网发布一份对阿里巴巴行政指导的“白皮书”,淘宝发布声明称工商总局网监司司长刘红亮“情绪执法”,决定向工商总局投诉,“白皮书”当晚被撤。1月30日,工商总局称“白皮书”不具有法律效力,总局局长会见马云,马云表示阿里巴巴将配合政府打假。\cite{Jing}\par
工商总局与阿里巴巴之间的互相指责以和解收场,然而已在美国上市的阿里巴巴遭到当地7家律师事务所的集体诉讼,被指控发布误导性声明并隐藏了收到监管调查的情况。1月26日收盘价103.99美元的阿里巴巴到3月13日已跌至81.86美元\cite{SinaNews},蒸发了约五分之一的市值,其中1月26日收盘价103.99美元,到1月30日时,收盘价为89.08美元,四个交易日总市值蒸发约367.53亿美元。\cite{Jing}上述诉讼到目前没有新进展。
\section{事件背后}
这个事件引起了国内外众多媒体和商界人士的关注,关注点一是阿里巴巴本身的影响力,二是引起这次官司的法律依据——Sarbanes-Oxley Act。2001年以来,上市公司和证券市场不断曝出丑闻,2002年6月的世界通信公司会计丑闻事件,“彻底打击了投资者对资本市场的信心”\cite{Wiki}。2002年美国国会通过了Sarbanes-Oxley Act,法案第一句:To protect investors by improving the accuracy and reliability of corporate disclosures made pursuant to the securities laws, and for other purposes.\cite{GPO}大意为:为了保护投资者和其他目的,改善公司信息披露的准确性和可靠性,遵守证券法律。Sarbanes-Oxley Act中被视为最重要的一款是第404款,要求“上市公司每年出具一份由独立董事签署的代表董事会的报告,说明该上市公司在财务管理和其它重要管理方面主要弊端在哪里。”\cite{Baidu}\par
根据该法案,阿里巴巴在上市前未能向美国有关方面披露曾经遭到工商总局调查等不利信息是违反该法案的行为。因此在此次阿里巴巴与工商总局的“骂战”之后,美国部分律师事务所找准机会,拉到一些阿里巴巴的小股东,集体起诉阿里巴巴。\par
中国在美上市企业受到多次围攻,包括中石油、新东方、分众传媒、兰亭集势、世纪互联、聚美优品、安博教育、龙威石油等等。\cite{Jing}
\section{事件点评}
该法案本意是加强对上市公司的监管,促使上市公司加强内部控制、信息披露等工作,保护投资者,尤其是中小投资者的合法权益。上市公司既然得到了在证券市场融资的权利,必然要承担相应的义务,主要是信息披露,一是在上市之前不隐瞒任何重大事项,二是在上市后定期或不定期披露信息,包括公司财报,人员重组等重大事项,以及该法案中提到的需要报告公司管理方面的不足和弊端。这种信息披露是非常必要的,一方面,投资者可以清楚的看到公司真实的运营状况,以自由选择是否投资,使投资者不被上市公司报喜不报忧的报告所欺骗;另一方面,重大事项及时公布可以有效防止暗箱操作,使中小股民免受损失。\par
然而法案也带来了一系列副作用,包括降低公司上市积极性、大幅提高公司管理和公关成本等等。由于法案要求对于希望上市的公司过于严厉,将增加许多公司在二级市场上融资的难度,另外为了满足该法案的要求,公司不得不花大量精力在公司内控和财务管理上,这将提高公司的管理成本。除此之外,违反法案的公司虽然通常会为此花费巨大代价,但是这种代价靠的是许多律师事务所,有部分律所专门针对上市公司经常上诉。上市公司的违法成本反而成为少数人的巨大收益,从另一个层面上造成了不公平。\par
美国的司法体系决定了上市公司一旦被诉,诉讼将是一个非常漫长和充满变数的过程,在这个过程中,上市公司往往要花费大量精力应诉,如果打赢官司则可以免遭惩罚,如果打输官司则面临巨额赔偿,因此许多被诉公司采取和解的办法,给律所和原告支付一笔费用并撤诉。在这个过程中,公司股价一般都会受到打压,遭受损失。\par
总的来说,该法案的出发点是好的,保护了投资者利益,可以有效避免2001年安然公司破产丑闻等造成许多公司职工和投资者经济损失的事件发生,但是在法案执行过程中,其副作用也体现出来,严厉的要求阻止了一些公司的上市,对于上市公司来说是一个沉重的负担,法案有利有弊。对于阿里巴巴等中国赴美上市的公司则好好上了一课,打铁还需自身硬,要想上市应当先管好自己,并且保护投资者合法权益不受侵犯。
\section{应对方法}
作为阿里巴巴的管理者,首要的任务是积极应诉,花重金聘请有经验、有资历的律师代为处理,尽快了解官司,最好双方达成和解。\par
第二步则是从中吸取教训。由于中国的证券市场不及美国成熟,监管不到位等现象时有发生,阿里巴巴等赴美上市的公司不了解美国对于上市公司的监管要求,放松了这一方面的工作。从这次事件以后,应当逐步适应美国证券市场法律,不要抱有侥幸心理。对于公司来说,及时公布不利信息造成的股价波动而产生的损失比事后被揭发产生的代价要小得多。\par
第三是建立合适的应急机制。此次与工商总局的“骂战”实质是应变措施失误的表现,“店小二”发微博指责工商总局直接导致工商总局将对阿里巴巴不利的更多信息披露出来。如果当时能够及时与工商总局联系协商,及时公开解释或道歉,事情都不会向更坏的方向发展。\par
根据之前一些公司的经历,此次事件以诉讼双方和解告终的可能性比较大,阿里巴巴为了早日摆脱诉讼的不利影响将倾向于支付大额赔偿平息本事件,律所为了经济利益也很可能接受和解,目前双方主要将对赔偿金额进行协商。\par
总而言之,阿里巴巴在体量逐渐膨胀的同时,应当注意公司自身的管理,提高应变能力,打铁还需自身硬。
\begin{thebibliography}{}

\bibitem[1]{Jing}阿里巴巴在美遭集体诉讼, http://www.bjnews.com.cn/finance/2015/02/\\03/352472.html
\bibitem[2]{SinaNews}新浪财经, http://stock.finance.sina.com.cn/usstock/quotes/BABA.html
\bibitem[3]{Wiki}维基百科词条:萨班斯-奥克斯利法案, http://zh.wikipedia.org/wiki/萨班斯-奥克斯利法案
\bibitem[4]{GPO}U.S. Government Publishing Office, http://www.gpo.gov/fdsys/pkg/PLAW\\-107publ204/html/PLAW-107publ204.htm
\bibitem[5]{Baidu}百度百科词条:塞班斯法案, http://baike.baidu.com/link?url=k3NQ8Vbq\\UgNNhHhOPJjBRqVpvzybI3u-J-0sC2KRArRaw8c6zaFjbB\_AF4Erv2JUy\\SwuBWohXNTYnfNH5K4Jtq

\end{thebibliography}
\end{document}