\documentclass[UTF8]{ctexart}
\usepackage{graphicx}
\usepackage{multirow}
\usepackage{booktabs}
\usepackage{indentfirst}
\setlength{\parindent}{2em}
\usepackage{color}
\definecolor{lbcolor}{rgb}{0.9,0.9,0.9}
\usepackage{listings}
\lstset{backgroundcolor=\color{lbcolor}}
\lstset{keywordstyle=\color[rgb]{0,0,1}}
\lstset{commentstyle=\color[rgb]{0.133,0.545,0.133}}
\lstset{stringstyle=\color[rgb]{0.627,0.126,0.941}}
\lstset{language=Matlab}
\lstset{numbers=left}
\lstset{breaklines=true}
\author{何舜成}
\title{系统工程导论第二次作业}
\begin{document}
\maketitle
\section{问题描述}
小明已经是一名本科三年级的学生了,面对大四即将到来的毕业选择,他常常在思考:出国,读硕,直博,还是直接工作?\par
在今天的系统工程课后,小明突然想到可以用AHP方法帮自己做一个决策。\par
对于上述4个毕业选择,小明有3个考虑的原则:\par
1)以自己的成绩和能力,作这个选择的难度如何;\par
2)从自己的性格和以往的经验来说,自己是否适合或者喜欢这个选择;\par
3)这几个选择对自己的职业发展影响如何。\par
对于第三个原则,小明认为过于宽泛,经深思熟虑,觉得这条原则可分为3小点考虑:\par
3.1)(毕业后)找工作的难度;\par
3.2)工作得到的待遇;\par
3.3)学位和履历对自己长期发展影响。\par
请同学们合理构想一个小明,简要描述他的基本情况。并利用AHP方法,替他为这四个选择排序,给出权重。\par

\section{参数确定}
四个选择:出国、读硕、直博、工作\par
三个原则:难度、喜爱程度、职业发展\par
原则三的子规则:找工作的难度、工作待遇、学历的长期影响\par
\begin{table}[!hbp]
\centering
\begin{tabular}{|c|c|c|c|}
\hline
\hline
 & 原则一 & 原则二 & 原则三 \\
\hline
原则一 & 1 & 1/5 & 1/4 \\
\hline
原则二 & 5 & 1 & 2 \\
\hline
原则三 & 4 & 1/2 & 1 \\
\hline
\end{tabular}
\caption{判断矩阵A:三个原则的排序}
\vspace{4mm}
\end{table}
\begin{table}[!hbp]
\centering
\begin{tabular}{|c|c|c|c|}
\hline
\hline
      & 规则一 & 规则二 & 规则三 \\
\hline
规则一 & 1     & 1/6   & 1/2   \\
\hline
规则二 & 6     & 1     & 4     \\
\hline
规则三 & 2     & 1/4   & 1     \\
\hline
\end{tabular}
\caption{判断矩阵A1:原则三子规则排序}
\vspace{4mm}
\end{table}
\begin{table}[!hbp]
\centering
\begin{tabular}{|c|c|c|c|c|}
\hline
\hline
      & 选择一 & 选择二 & 选择三 & 选择四 \\
\hline
选择一 & 1     & 1/3   & 5     & 1     \\
\hline
选择二 & 3     & 1     & 7     & 3     \\
\hline
选择三 & 1/5   & 1/7   & 1     & 1/8   \\
\hline
选择四 & 1     & 1/3   & 8     & 1     \\
\hline
\end{tabular}
\caption{判断矩阵B1:四个选择相对于原则三子规则一(找工作难度)的排序}
\vspace{4mm}
\end{table}
\begin{table}[!hbp]
\centering
\begin{tabular}{|c|c|c|c|c|}
\hline
\hline
      & 选择一 & 选择二 & 选择三 & 选择四 \\
\hline
选择一 & 1     & 4     & 1/2   & 7     \\
\hline
选择二 & 1/4   & 1     & 1/4   & 4     \\
\hline
选择三 & 2     & 4     & 1     & 9     \\
\hline
选择四 & 1/7   & 1/4   & 1/9   & 1     \\
\hline
\end{tabular}
\caption{判断矩阵B2:四个选择相对于原则三子规则二(工作待遇)的排序}
\vspace{4mm}
\end{table}
\begin{table}[!hbp]
\centering
\begin{tabular}{|c|c|c|c|c|}
\hline
\hline
      & 选择一 & 选择二 & 选择三 & 选择四 \\
\hline
选择一 & 1     & 4     & 1/2   & 7     \\
\hline
选择二 & 1/4   & 1     & 1/6   & 4     \\
\hline
选择三 & 2     & 6     & 1     & 9     \\
\hline
选择四 & 1/7   & 1/3   & 1/9   & 1     \\
\hline
\end{tabular}
\caption{判断矩阵B3:四个选择相对于原则三子规则三(学历的长期影响)的排序}
\vspace{4mm}
\end{table}
\begin{table}[!hbp]
\centering
\begin{tabular}{|c|c|c|c|c|}
\hline
\hline
      & 选择一 & 选择二 & 选择三 & 选择四 \\
\hline
选择一 & 1     & 1/5   & 3     & 1/3   \\
\hline
选择二 & 5     & 1     & 7     & 3     \\
\hline
选择三 & 1/3   & 1/7   & 1     & 1/8   \\
\hline
选择四 & 3     & 1/3   & 8     & 1     \\
\hline
\end{tabular}
\caption{判断矩阵C1:四个选择相对于原则一(难度)的排序}
\vspace{4mm}
\end{table}
\begin{table}[!hbp]
\centering
\begin{tabular}{|c|c|c|c|c|}
\hline
\hline
      & 选择一 & 选择二 & 选择三 & 选择四 \\
\hline
选择一 & 1     & 6     & 3     & 4     \\
\hline
选择二 & 1/6   & 1     & 1/3   & 1/3   \\
\hline
选择三 & 1/3   & 3     & 1     & 1     \\
\hline
选择四 & 1/4   & 3     & 1     & 1     \\
\hline
\end{tabular}
\caption{判断矩阵C2:四个选择相对于原则二(喜爱程度)的排序}
\vspace{4mm}
\end{table}
\section{运行结果}
7个矩阵的C.R.分别为:\par
\begin{table}[!hbp]
\centering
\begin{tabular}{|c|c|c|c|c|c|c|c|}
\hline
\hline
      & 矩阵A  & 矩阵A1 & 矩阵B1 & 矩阵B2 & 矩阵B3 & 矩阵C1 & 矩阵C2 \\
\hline
C.R.  & 0.0212 & 0.0079 & 0.0550 & 0.0395 & 0.0071 & 0.0582 & 0.0170 \\
\hline
\end{tabular}
\caption{7个矩阵的C.R.}
\vspace{4mm}
\end{table}
均符合一致性要求。\par
最后得到的权重-得分矩阵为:\par
\begin{table}[!hbp]
\centering
\begin{tabular}{@{}ccccccc@{}}
\toprule
 & \multirow{2}{*}{难度} & \multirow{2}{*}{喜爱程度} & \multicolumn{3}{c}{职业发展影响} & \multirow{4}{*}{总得分} \\
 & & & \multicolumn{3}{c}{0.3331} & \\
\cline{4-6}
 & \multirow{2}{*}{0.0974} & \multirow{2}{*}{0.5695} & 找工作难度 & 工作待遇 & 学历长期影响 & \\
 & & & 0.1061 & 0.7010 & 0.1929 & \\
\midrule
出国 & 0.1111 & 0.5609 & 0.2019 & 0.3354 & 0.3203 & 0.4363 \\
读硕 & 0.5546 & 0.7010 & 0.5176 & 0.1227 & 0.0991 & 0.1477 \\
直博 & 0.0482 & 0.1898 & 0.0456 & 0.4988 & 0.5376 & 0.2654 \\
工作 & 0.2861 & 0.1783 & 0.2349 & 0.0431 & 0.0431 & 0.1505 \\
\bottomrule
\end{tabular}
\caption[]{最终结果}
\vspace{4mm}
\end{table}
依照此结果,可以有较大的把握认为选择出国是四个选择中最好的。\par
\section{具体实现}
Matlab代码如下所示:\par
\begin{lstlisting}
%%Codes:
clc;
clear all;
A = [1 1/5 1/4; 5 1 2; 4 1/2 1];
A1 = [1 1/6 1/2; 6 1 4; 2 1/4 1];
B1 = [1 1/3 5 1; 3 1 7 3; 1/5 1/7 1 1/8; 1 1/3 8 1];
B2 = [1 4 1/2 7; 1/4 1 1/4 4; 2 4 1 9; 1/7 1/4 1/9 1];
B3 = [1 4 1/2 7; 1/4 1 1/6 3; 2 6 1 9; 1/7 1/4 1/9 1];
C1 = [1 1/5 3 1/3; 5 1 7 3; 1/3 1/7 1 1/8; 3 1/3 8 1];
C2 = [1 6 3 4; 1/6 1 1/3 1/3; 1/3 3 1 1; 1/4 3 1 1]; 

%%Step 1: Calculate the weights of the three principles
[x, lambda] = eig(A);
r = abs(sum(lambda));
n = find(r==max(r));
max_lambda = lambda(n,n);
max_x = x(:,n);
sum_x = sum(max_x);
weight_A = max_x/sum_x;

cr = (max_lambda-3)/2/0.58;
disp(cr);
pause();

%%Step 2: Calculate the weights of the three subrules
[x, lambda] = eig(A1);
r = abs(sum(lambda));
n = find(r==max(r));
max_lambda = lambda(n,n);
max_x = x(:,n);
sum_x = sum(max_x);
weight_A1 = max_x/sum_x;

cr = (max_lambda-3)/2/0.58;
disp(cr);
pause();

%%Step 3: Calculate the score of the first subrule
[x, lambda] = eig(B1);
r = abs(sum(lambda));
n = find(r==max(r));
max_lambda = lambda(n,n);
max_x = x(:,n);
sum_x = sum(max_x);
score_B1 = max_x/sum_x;

cr = (max_lambda-4)/3/0.90;
disp(cr);
pause();

%%Step 4: Calculate the score of the second subrule
[x, lambda] = eig(B2);
r = abs(sum(lambda));
n = find(r==max(r));
max_lambda = lambda(n,n);
max_x = x(:,n);
sum_x = sum(max_x);
score_B2 = max_x/sum_x;

cr = (max_lambda-4)/3/0.90;
disp(cr);
pause();

%%Step 5: Calculate the score of the third subrule
[x, lambda] = eig(B3);
r = abs(sum(lambda));
n = find(r==max(r));
max_lambda = lambda(n,n);
max_x = x(:,n);
sum_x = sum(max_x);
score_B3 = max_x/sum_x;

cr = (max_lambda-4)/3/0.90;
disp(cr);
pause();

%%Step 6: Calculate the score of the first principle
[x, lambda] = eig(C1);
r = abs(sum(lambda));
n = find(r==max(r));
max_lambda = lambda(n,n);
max_x = x(:,n);
sum_x = sum(max_x);
score_C1 = max_x/sum_x;

cr = (max_lambda-4)/3/0.90;
disp(cr);
pause();

%%Step 7: Calculate the score of the second principle
[x, lambda] = eig(C2);
r = abs(sum(lambda));
n = find(r==max(r));
max_lambda = lambda(n,n);
max_x = x(:,n);
sum_x = sum(max_x);
score_C2 = max_x/sum_x;

cr = (max_lambda-4)/3/0.90;
disp(cr);
pause();

%%Step 8: Calculate the final result
s1 = score_B1*weight_A(3)*weight_A1(1);
disp(s1);
s2 = score_B2*weight_A(3)*weight_A1(2);
disp(s2);
s3 = score_B3*weight_A(3)*weight_A1(3);
disp(s3);
s4 = score_C1*weight_A(1);
disp(s4);
s5 = score_C2*weight_A(2);
disp(s5);

score = s1+s2+s3+s4+s5;
disp(score);
\end{lstlisting}
\end{document}