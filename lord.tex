\documentclass[UTF8]{ctexart}
\usepackage{indentfirst}
\setlength{\parindent}{2em}
\usepackage{cite}
\author{自26 何舜成}
\title{电子商务环境下企业管理面临的主要问题}
\begin{document}
\maketitle
\tableofcontents
\section{电子商务大环境}
随着计算机、通信、网络技术的高速发展,越来越多的企业开始融入到电子商务大环境中来,建立信息化系统,搭建电子商务平台。在信息化融入国民经济各个领域的今天,电子商务给企业带来了发展的新机遇。通过电子商务平台,企业可以提高信息沟通的效率,通过网络进行市场调研,发布新产品,消费者或下游企业可以通过网络订购商品;企业可以降低运营成本,方便快捷地进行商务谈判,降低信息搜集和交换的成本;企业可以更好地与市场沟通,生产者和消费者之间的双向互动将更加快捷。电子商务提供了一个信息流、资金流、物流的快速通道,将极大地改变市场的生态环境和企业的组织面貌。\par
国家“十二五”规划中提出推动信息化和工业化的融合发展,积极发展电子商务。\cite{Plan}而电子商务也成为原来越多的企业的选择。在这个大环境下,电子商务已经成为政策导向和市场潮流,未来的企业要向信息化方向发展。企业信息化应当成为企业战略层面上的布局,电子商务应当成为企业经营的得力工具。\par
然而在企业信息化与电子商务的大浪潮下,大型企业应当如何转型,如何转变管理思路,小型企业应当如何生存,如何借力电子商务?在这机遇之中企业还面临着种种挑战,如何应对是一个非常值得关注的问题。\par

\section{电子商务环境下的挑战}
\subsection{大型企业面临的挑战}
电子商务包含B2B、B2C、C2C几种,C2C一般指网上交易平台如淘宝、京东等,而更多的大型企业需要着重于发展B2B、B2C业务,利用B2B与其它大型企业进行合作,解决供应链上下游互联互通的问题,如1999年通用和福特两大汽车制造商将各自的采购部门转移到互联网上,在全球范围内进行原料采购;\cite{Book}利用B2C与消费者之间实现畅通无阻的交流和沟通,大部分电子产品公司如苹果公司就利用公司网站进行产品的展示和发布,消费者通过网站购买商品,获得产品支持,及时反馈用户体验。B2B、B2C平台建设成为大型企业发展电子商务的重要任务。\par
然而,在建设B2B、B2C平台时,大型企业面临许多挑战。一是企业自身信息化问题,企业内部的信息化水平是否足够高,这影响着电子商务平台的运转是否能提高效率、降低成本,真正为我所用;二是电子商务平台的管理问题,运营维护队伍是否专业,平台的稳定性、安全性是否能得到保障,用户体验是否良好也是非常重要的。\par
由于电子商务平台速度快的特性,大型企业由于体量大、结构复杂,往往需要加强内部的信息化管理。举例来说,客户在网上订购了某种产品,系统在确认订单后需要及时将信息传送至仓储部门,仓储部门收到发货要求后及时将货物递交物流部门或委托第三方物流公司,仓储部门再将库存信息反馈到电子商务平台。在这个过程中,各个部门之间的信息互联和共享是非常重要的。企业实行信息化之后,一切电子商务平台上的交易信息可以迅速地反馈至各个部门,各个部门自动地做出响应,使得企业内部运转的速度可以平台的速度同步。否则电子商务平台无法实质性提高效率,反而增加管理成本。\par
大型企业构建电子商务平台还需要考虑平台的运营维护,企业需要专业的管理团队。平台网站需要与客户打交道,尤其B2C、C2C平台的用户友好程度甚至可以影响销售业绩。大型企业平台上数据量多,交易量大,这时往往要保证平台的稳定性,避免数据丢失、被窃。这些都需要一个专业的管理团队,如何建立这样一支队伍也是大型企业的一大挑战。
\subsection{小型企业面临的挑战}
电子商务环境下小型企业面临的挑战则不同,小型企业往往资金少、融资难、人才少、市场影响力弱,在这样的基础建立一整套电子商务系统是不现实的。\cite{Xu}因此小型企业需要另辟蹊径,创新电子商务发展模式。\par
当前一些小型企业请专业的网络技术公司建立独立的电子商务网站,但网站的运营成本高,网站的推广也十分困难。网站建立之后需要购置一批硬件设施(如服务器)以应对大量的数据存储和出入站流量;网站的维护更新需要有专门的运营团队,成熟的运营团队开销也让小企业承受不了;要想达到规模效应,网站的浏览量、知名度都需要提升,需要投入大量广告和推广费用,这些也不是普通小型企业可以做到的。\cite{Yue}因此许多小型企业虽然建立起了自己的电子商务平台,但是并没有享受到规模效应带来的发展机遇,电子商务平台反而成为一些企业的负担。\par
物流也是小型企业发展电子商务的一大难题。大多数小型企业将商品配送外包给第三方物流,但是这种模式也存在一些问题。物流往往是交易过程中影响客户满意度的一个环节,而这个环节往往是小型企业自己所无法掌控的。企业必须挑选服务品质好、配送速度快的物流公司以提高客户满意度,这个过程中小型企业又要投入较高的成本。\par
总的来说,小型企业发展电子商务的两大瓶颈是网络和物流。由于小型企业没有足够的资金去支持建立一套独立的电商平台和物流服务,只能将这两大部分分别外包给服务提供商。但是这种方案成本并不低,效果也并不一定显著。

\section{企业如何应对}
\subsection{大型企业的管理思路}
对于大型企业首要的任务是抓好企业内部的信息化,需要有一个长远的计划和考虑,将企业信息化上升至全体领导层的共识,通盘规划、全面协调推进信息化建设,将电子商务系统的建设纳入企业信息化建设中来。国内外优秀的大型企业都不只是有一个高效的电子商务系统,而是企业的组织机构、管理制度、运作模式全方位的革新,电子商务只是其中的一部分。针对上述大型企业在电子商务环境下的种种挑战,可以提出以下几点管理改革的思路:\par
一是将信息化作为企业意志。企业信息系统建成后将影响到大部分员工的工作,也可能涉及到不同部门和岗位之间的利益冲突,因此需要高层将企业信息化作为一个系统工程来考虑。企业应当协调好各部门之间的利益,做好一些岗位的培训学习工作、使用绩效考核等方式引导员工逐步适应信息化的工作。\par
二是协调各个部门之间的关系,以企业信息系统为平台,加强部门间的横向联系。在信息系统建立之后,企业组织结构可能更加趋于扁平化,同一层级之间横向的信息流动将增强,这是为了缩短消息和指令的传递路径,减少长路径带来的延迟,一方面提高沟通效率,另一方面减少沟通成本。部门间横向联系加强以后,企业应对外来变化将更加敏捷灵活,提高竞争力。\par
三是建立一支专业的信息化(电子商务)人才队伍。信息化不是一蹴而就的事,需要在实践中不断调整和改良,以适应客户以及企业内部发展的需要。因此在信息系统和电子商务平台建立好之后需要大量的精力来维护和调整。做这个工作需要一支专业队伍,既了解各部门的运作模式和组织架构,又对市场和客户需求敏感,还要有专项的技术技能。这样才能使电子商务系统发挥最大限度的优势。
\subsection{小型企业的管理思路}
对于小型企业来说重要的是要创新电子商务模式,发挥规模效应。有学者提出“公用信息平台”\cite{Yue}和“第四方物流”\cite{Wiki}之说,“公用信息平台”是各类生产商、消费者、物流公司的集散地,小型企业以会员的身份加入该平台,由平台来负责提升访问量和知名度,负责日常的运营和维护,而费用分摊到各个作为会员的小型企业。“第四方物流”则是具有市场经验和专业资质的公司,负责为小企业量身定制高效率、个性化、低成本的物流解决方案,由于“第四方物流”可以为许多小型企业提供服务,该项服务的成本也将由众多小型企业共同承担。\par
无论是“公用信息平台”还是“第四方物流”都是规模经济的体现。小型企业资金、人才各方面都缺乏,抱团取暖是一个比较好的选择,成本公摊,成果共享,由专门的电子商务公司(第三方电子商务平台)为小型企业提供成熟的解决方案。\par
在这种环境下,小型企业还需要做好与第三方电子商务平台的接口,实现与第三方电子商务平台在信息流、资金流、物流上的无缝对接,发挥小型企业灵活性强的优势,利用互联网及时把握市场动向,把握消费者需求,让电子商务平台发挥更大的作用。

\section{总结与展望}
电子商务是一种必然的趋势,无论大型企业还是小型企业都可以抓住机遇,让电子商务助力于自身的发展,提高企业运营效率,降低交易成本,打开更广阔的市场。然而在实施电子商务战略过程中,大型企业和小型企业又面临着不同的挑战。大型企业体量巨大,难以适应电子商务的快节奏、大数据,因此大型企业需要从整体考虑建立企业自身的信息化系统,打通企业内部的信息障碍,然后培养一支专业的信息化队伍,负责电子商务系统的建设和维护。小型企业资金少、人才不足,无法建立独立的电子商务系统,而如果仅仅将电子商务系统交给第三方建设又面临难以打开市场、成本降不下来的困难,因此小型企业需要“抱团取暖”,发挥规模效应,由专业的“公用信息平台”和“第四方物流”替众多小型企业提供解决方案。事实上,“公用信息平台”的概念已经有了体现,比如淘宝、京东等网络购物平台就为一些小型服装企业、食品加工企业提供了一个与消费者互动、交易的平台,而“第四方物流”公司也有所发展,其提出者美国的埃森哲咨询公司就是一家整合物流资源,为第一方物流、第二方物流、第三方物流提供规划、咨询、管理服务。\cite{Wiki}电子商务的发展方兴未艾,未来还将极大地改变企业与企业、企业与消费者之间的交易方式,更代表着企业信息化的大方向。

\begin{thebibliography}{}
\bibitem[1]{Plan}“十二五”规划纲要全文,中国网,http://www.china.com.cn/policy/\
txt/2011-03/16/content\_22156007\_4.htm
\bibitem[2]{Book}范玉顺,信息化管理战略与方法,清华大学出版社,p362-367
\bibitem[3]{Xu}徐莹,刘婷,第三方电子商务平台——中小企业发展的最佳之选,中国集体经济,2007(08)
\bibitem[4]{Yue}乐奕平,王辉球,我国中小企业电子商务发展新模式,商业研究,2004(22)
\bibitem[5]{Wiki}第四方物流,维基百科词条,http://zh.wikipedia.org/wiki/第四方物流
\end{thebibliography}
\end{document}