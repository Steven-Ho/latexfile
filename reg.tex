\documentclass[UTF8]{ctexart}
\usepackage{graphicx}
\usepackage{multirow}
\usepackage{booktabs}
\usepackage{latexsym}
\usepackage{indentfirst}
\setlength{\parindent}{2em}
\usepackage{color}
\definecolor{lbcolor}{rgb}{0.9,0.9,0.9}
\usepackage{listings}
\lstset{backgroundcolor=\color{lbcolor}}
\lstset{keywordstyle=\color[rgb]{0,0,1}}
\lstset{commentstyle=\color[rgb]{0.133,0.545,0.133}}
\lstset{stringstyle=\color[rgb]{0.627,0.126,0.941}}
\lstset{language=Matlab}
\lstset{numbers=left}
\lstset{breaklines=true}
\author{何舜成,厉丹阳}
\title{利用摄动理论求解跨步方程的近似解析表达}
\begin{document}
\maketitle
\section{摄动理论简介}
在一些很难求得精确解的数学方程中,引入微扰项,从而求得近似的解析解,这种方法称作摄动方法或摄动理论(Perturbation Theory)。下面是一个简单的运用摄动理论求解方程的例子。\par
考虑如下二次方程\par
\begin{equation}
x^{2}-2\epsilonx-1=0, \epsilon\ll1
\end{equation}
\begin{equation}
F=\frac{(N-n+r-1)ESS}{(n-r)RSS}
\end{equation}
\section{计算思路}
假设$X$是$N \times n$的输入样本矩阵,$x_{i}$表示第$i$列样本,$Y$是$N \times 1$的输出样本矩阵,首先对样本进行归一化:\par
\begin{equation}
\tilde{x_{i}}=\frac{x_{i}-\bar{x_{i}}}{\sigma_{x_{i}}}, \forall i=1,2,...n
\end{equation}
\begin{equation}
\tilde{Y}=\frac{Y-\bar{Y}}{\sigma_{Y}}
\end{equation}
\par
接下来进行病态分析,求取$X^{T}X$的特征值,并进行正交分解:\par
\begin{equation}
L=X^{T}X
\end{equation}
\begin{equation}
UTU^{T}=L, s.t.UU^{T}=I
\end{equation}
\par
剔除接近0的特征值($r$个)以及相对应的特征向量得到$U_{m}$和$T_{m}$,维度降到了$n-r$,进行线性回归,求得数据归一化、中心化之后的拟合系数:\par
\begin{equation}
\bar{c}=U_{m}T^{-1}_{m}U^{T}_{m}X^{T}Y
\end{equation}
\par
对该拟合系数作变换,并计算截距($\bar{X}$表示$X$每一列取平均得到的行向量):\par
\begin{equation}
c_{i}=\frac{\bar{c_{i}}\sigma_{Y}}{\sigma_{x_{i}}}, \forall i=1,2,...n
\end{equation}
\begin{equation}
\beta_{0}=\bar{Y}-\bar{X}c
\end{equation}
\par
对结果进行$F$检验(考虑病态情形):\par
\begin{equation}
F=\frac{(N-n+r-1)ESS}{(n-r)RSS}
\end{equation}
\par
计算置信区间:\par
\begin{equation}
d=Z_{\frac{\alpha}{2}}\sqrt{\frac{RSS}{N-n+r-1}}
\end{equation}
\section{计算结果}
以下检验均在显著性水平0.05下进行。\par
回归方程及置信区间:\par
\begin{equation}
y=-9.151455+0.072966x_{1}+0.598562x_{2}+0.001872x_{3}+0.105482x_{4}\pm1.155715
\end{equation}
\par
$F$检验结果表明可认为线性关系成立:\par
\begin{equation}
F=195.116494>F_{\alpha}=4.346831
\end{equation}
\section{具体实现}
Matlab代码(.m文件)如下所示:\par
\begin{lstlisting}
% function head
function linear_regression(y,x,alpha)
% set sample average to 0, 
% and set sample standard deviation to 1
Y = transpose(y);
Y = (Y - mean(Y))/std(Y);
[row,col] = size(x);
X = x;
avgX = zeros(1,col);
stdX = zeros(1,col);
for k = 1:col
    c = x(:,k);
    avgX(1,k) = mean(c);
    stdX(1,k) = std(c);
    X(:,k) = (c - mean(c))/std(c);
end
% attach a column of 1 to matrix x
% as the interception
ln = ones(row,1);
Xt = [ln,X];
X = transpose(Xt);
% decomposition
L = X*Xt;
[U,T] = schur(L);
% rule out small eignvalues and vectors
m = 0;
for k = 1:(col+1)
    if T(k,k) > 0.1
       m = m + 1; 
    end
end
Qm = zeros(col+1,0);
Vm = zeros(m,m);
i = 1;
for k = 1:(col+1)
    if T(k,k) > 0.1
       Qm = [Qm,U(:,k)];
       Vm(i,i) = T(k,k);
       i = i + 1;
    end
end
% solve for coefficients
cbar = std(y)*Qm*inv(Vm)*transpose(Qm)*X*transpose(Y);
yhat = Xt*cbar + mean(y);
for k = 1:col
   cbar(k+1,1) = cbar(k+1,1)/stdX(k);
end
cbar(1,1) = mean(y) - avgX*cbar(2:col+1,1);
% F test
TSS = dot(y-mean(y),y-mean(y));
ESS = dot(yhat-mean(y),yhat-mean(y));
RSS = TSS - ESS;
F = (row-m)*ESS/((m-1)*RSS);
Fa = finv(1-alpha,m-1,row-m);
%fprintf('F = %f, Fa = %f\n',F,Fa);
if F<=Fa
%    fprintf('No linear relation! (a=%.2f)\n',alpha);
else
%    fprintf('Linear relation! (a=%.2f)\n',alpha);
end
% solve for confidence interval
sdelta = sqrt(RSS/(row-m));
Z = norminv(1-alpha/2,0,1) * sdelta;
%fprintf('Confidence interval: -%f ~ +%f\n',Z,Z);
%fprintf('Regression equation: %f + %f*x1 + %f*x2 + %f*x3 + %f*x4\n',...
%cbar(1,1),cbar(2,1),cbar(3,1),cbar(4,1),cbar(5,1));
end
% function ending
\end{lstlisting}
\end{document}